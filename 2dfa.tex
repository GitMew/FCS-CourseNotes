\documentclass{seminar}
%==============================================================================
\usepackage{fancybox}
\usepackage{semcolor}
\usepackage{semlayer}
\usepackage{sem-a4}
\usepackage{epsfig}
\usepackage{subfigure}
\usepackage{amssymb}
%==============================================================================
%COLORS: black,darkgray,gray,lightgray,white,red,green,blue,yellow,cyan,magenta
%==============================================================================
\newcommand{\LG}{\lightgray}
%==============================================================================
% For Black and White slides uncomment the following
%---------------------------------------------------
\renewcommand{\red}{\black}
\renewcommand{\blue}{\black}
\renewcommand{\gray}{\black}
\renewcommand{\green}{\black}
\renewcommand{\magenta}{\black}
\renewcommand{\darkgray}{\black}
\newcommand{\beginm}{\vdash }
\newcommand{\eindm}{\dashv }
\newcommand{\prend}{\hfill \rule{2.3mm}{2.3mm} \hspace*{3mm} $ $}



% ==============================================================================
%==============================================================================

\def\printlandscape{\special{landscape}}    % Works with dvips.

% \slideframe{Oval}

% \onlyslides{1-30}

\overlaystrue
\printlandscape


\begin{document}

\begin{slide}

{\bf Equivalentie van DFA en 2-DFA}

\begin{itemize}
\item 
definitie van 2-DFA

\item 
een equivalentierelatie gebaseerd op een 2-DFA voor een L

\item
het is een MN(L)-relatie !


\end{itemize}

Snel opgeschreven op 20 november 2011 door Bart Demoen

\end{slide}  

\begin{slide}

{\bf Definitie van 2-DFA}

Informeel: 2-DFA kan leeskop links en rechts bewegen ...

\begin{itemize}
\item 
($Q$, $\Sigma$, $\delta$, $\beginm$, $\eindm$, $q_s$, $q_a$, $q_r$)
\begin{itemize}
\item 
$\beginm$~en $\eindm$~niet in $\Sigma$ (linkse en rechtse marker)

\item 
$\delta$:
$(Q \times (\Sigma \cup \{\beginm, \eindm\})) \rightarrow Q \times\{L,R\}$



\item 
$\{q_s, q_a, q_r\} \subseteq Q$ (start, accept en reject toestand)
\item
$\#\{q_s, q_a, q_r\} = 3$

\end{itemize}


\item 
niet te links of te rechts lezen ...
\begin{itemize}
\item 
$\forall p, \exists q: \delta(p, \beginm) = (q \times R)$
en
$\forall p, \exists q: \delta(p, \eindm) = (q \times L)$

\end{itemize}


\item 
bij accept/reject: leeskop helemaal naar rechts
\begin{itemize}
\item 
$\delta(q_a, \_) = (p \times R)$ en $\delta(q_r, \_) = (p \times R)$

\end{itemize}


\end{itemize}

\end{slide} 

\begin{slide}

{\bf String s geaccepteerd door 2-DFA}

\begin{itemize}
\item
start 2-DFA in $q_s$ op string $\beginm s\eindm$

\item 
geaccepteerd als de 2-DFA in de situatie $(q_a, \eindm)$ komt

s behoort tot de taal van de 2-DFA
\item 
verworpen als de 2-DFA in de situatie $(q_r, \eindm)$ komt

s behoort niet tot de taal van de 2-DFA

\item 
anders ... 

s behoort niet tot de taal van de 2-DFA

\end{itemize}

2-DFA mag in een lus gaan en is dus een beetje zoals een TM

strings waarvoor de 2-DFA in een lus gaat behoren niet tot de taal
bepaald door de 2-DFA

\end{slide} 

\begin{slide}

{\bf Overtuig jezelf ...}

dat de {\em speciale} eisen niet {\em belangrijk} zijn (wat betekent
dat juist in deze context ?)

maar ze zullen het ons wel iets gemakkelijker maken

\end{slide} 

\begin{slide}

{\bf Het plan}

Gegeven een 2-DFA die taal L bepaalt:

\begin{itemize}
\item 
we defini\"eren een equivalentierelatie $\equiv$ op $\Sigma^*$

\item 
we bewijzen dat $\equiv$ een MN(L)-relatie is

\item 
dat bewijst dat L regulier is

\end{itemize}


vanaf nu is de 2-DFA (en L) gegeven ...

\end{slide} 



\begin{slide}

{\bf Voorbereiding van $\equiv$ - deel I}

Voor een gegeven string $s$ maken we een functie

~~~~~~~~~~~~~~~~~$T_s: (Q \cup {\bullet}) \rightarrow (Q \cup {\bot})$

\begin{itemize}
\item[$T_s(\bullet)$]

\begin{itemize}
\item 
start 2-DFA op $\beginm s \eindm$ met leeskop op $\beginm$ in $q_s$

\item 
de eerste keer dat de leeskop op $\eindm$ komt, is de 2-DFA in
toestand $q$, dan $T_s(\bullet) \triangleq q$

\item 

als er zo geen eerste keer is: $T_s(\bullet) \triangleq \bot$
\end{itemize}



\item[$T_s(p)$]
\begin{itemize}
\item
start 2-DFA op $\beginm s \eindm$ met leeskop op symbool vlak voor $\eindm$ 
in toestand $p$
\item
de eerste keer dat de leeskop daarna op $\eindm$ komt, is de 2-DFA in
toestand $q$, dan $T_s(p) \triangleq q$

\item

als er zo geen eerste keer is: $T_s(p) \triangleq \bot$

\end{itemize}

\end{itemize}



\end{slide} 

\begin{slide}

{\bf Voorbereiding van $\equiv$: deel II}

Wat betekent die $T_s$ ?

\begin{itemize}
\item 
volg de werking van de 2-DFA op een string van de vorm $sw$

\item 
elke keer dat de leeskop van $w$ naar $s$ gaat en daar aankomt in
toestand $p$

\item 
dan: de eerste keer dat de leeskop daarna van $s$ naar $w$ gaat, komt
ie daar toe in $T_s(p)$ tenzij dat nooit gebeurt

\item
plus een speciaal geval voor $T_s(\bullet)$ want dat is een overgang van
$s$ naar $w$ waarvoor geen vorige overgang van $w$ naar $s$ was

\end{itemize}

$T_s$ bevat alle info die van $s$ naar $w$ kan vloeien


\end{slide} 

\begin{slide}
{\bf Interludium over $T_s$}

\begin{itemize}
\item 
als je wil beslissen of string $sw$ geaccepteerd wordt, dan heb je
genoeg aan
\begin{itemize}
\item 
$T_s$
\item 
$w$
\item 
$\delta$
\end{itemize}


\item maar $s$ moet je niet kennen

\end{itemize}

Lees niet verder totdat je dat verstaat !

Hint: speel zelf voor 2-DFA en kijk wat je nodig hebt als die een
overgang wil maken van $w$ naar $s$

\end{slide} 

\begin{slide}

{\bf Voorbereiding van $\equiv$: deel III}

hoeveel verschillende functies $(Q \cup {\bullet}) \rightarrow (Q \cup {\bot})$ bestaan er ?

\begin{itemize}
\item 
stel $\#Q = k$, dan $(k+1)^{(k+1)}$

\item 
juiste aantal niet belangrijk, wel dat het eindig is !

\end{itemize}


\end{slide} 

\begin{slide}
{\bf Definitie van $\equiv$ op $\Sigma^*$}

\begin{itemize}
\item 
$s \equiv t \Leftrightarrow T_s = T_t$

\item 
bewijs zelf dat dit een equivalentierelatie is

\item 
de ge\"{i}nduceerde partitie is eindig, want ...

\item
als we nu nog kunnen bewijzen dat 
\begin{itemize}
\item

die partitie $\{L, \overline{L}\}$
verfijnt en 

\item
dat $\equiv$ rechts-congruent is ...
\end{itemize}

{\bf Dan is $L$ regulier !}
\end{itemize}


\end{slide} 

\begin{slide}

{\bf $\equiv$ verfijnt $\{L, \overline{L}\}$: bewijs}

Stel $T_s = T_t$ en $s \in L$

\begin{itemize}
\item 
de overgangen van $s$ naar $\eindm$ en die van $t$ naar $\eindm$
gebeuren in dezelfde toestanden en in dezelfde volgorde (waarom ?)

\item 
laatste overgang voor $s$ is in toestand $q_a$

\item
dus ook voor $t$

\end{itemize}

dus ook $t \in L$

M.a.w. $T_s = T_t \Leftrightarrow (s \in L) \leftrightarrow (t \in L)$

of

~~~~~~~~~~~~~~~~~$\equiv$ verfijnt $\{L, \overline{L}\}$

\prend

De Interludium slide voordien kan je ook helpen ...
\end{slide} 


\begin{slide}
{\bf $\equiv$ is rechts-congruent: bewijs}

Stel $s \equiv t$, dan $T_s = T_t$ (per definitie)

Neem een $a \in \Sigma$

\begin{itemize}
\item 
$T_{sa}(\bullet)$ ($T_{ta}(\bullet)$) krijg je door
\begin{itemize}
\item 
stel $p = T_s(\bullet)~~(= T_t(\bullet))$

\item 
of: de 2-DFA komt de 1$^{ste}$ keer met leeskop boven de $a$ in $p$

\item 
$T_s = T_t$ en de overgangen tussen $s$ en $a$ gebeuren in dezelfde
volgorde als tussen $t$ en $a$, dus $T_{sa}(\bullet) =
T_{ta}(\bullet)$

\end{itemize}

\item 
start 2-DFA met leeskop onder $a$
en in een willekeurige toestand $p$
\begin{itemize}
\item 
leeskop gaat naar rechts ... $T_{sa}(p) = T_{ta}(p)$
\item 
of leeskop gaat naar links ... dus $T_{sa}(p) = T_{ta}(p)$
\end{itemize}

\end{itemize}

\prend

\end{slide} 


\begin{slide}
{\bf Besluit}

\begin{itemize}
\item 
elke taal L die bepaald wordt door een 2-DFA is regulier
\item 
elke taal L die bepaald wordt door een DFA wordt ook door een 2-DFA
bepaald
\item 
2-DFA en DFA zijn even {\em krachtig}
\end{itemize}

\end{slide} 

\begin{slide}
{\bf Om over na te denken ...}

\begin{itemize}
\item 
kan een 2-DFA voor een L kleiner zijn dan de minimale DFA ?

\item 
bestaat een minimalisatie-algoritme voor 2-DFA's ?

\item
hoe zit het met een transformatie van 2-DFA naar DFA ?

\item 
bestaan er ook 2-NFA's ? en welke eigenschappen hebben die ?

\item 
is $H_{2-DFA}$ herkenbaar, beslisbaar ?


\end{itemize}


\end{slide} 

\begin{slide}
{\bf Geschiedenis}

\begin{itemize}
\item 
probleem eerst bestudeerd (en opgelost) door M. Rabin en D. Scott (1959):
ingewikkeld bewijs
\item 
J. Shepherdson gaf al snel een eenvoudiger bewijs: het staat essentieel
hierboven - Rabin en Scott vinden het zelfs niet de moeite om hun
eigen bewijs in detail te publiceren, maar verwijzen naar Shepherdson
\item 
1989: M. Vardi geeft een nog {\em gemakkelijker} bewijs - ik vind het moeilijker :-(
\item
zie boek {\em Automata and Computability} van Dexter C. Kozen
\end{itemize}

\end{slide} 

\end{document}
