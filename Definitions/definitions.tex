\chapter*{Prerequisites}

There are several basic mathematical notions used
throughout this course that you should be familiar with.
These are briefly summarized here as a refresher.

%-------------------------------------------------------------------------------
\section*{Sets, Relations and Functions}

%- - - - - - - - - - - - - - - - - - - - - - - - - - - - - - - - - - - - - - - - 
\subsection*{Cartesian Products}
The Cartesian product of two sets $A$ and $B$, which is denoted $A \times B$ is
the set of all tuples that can be formed by pairing an element of $A$ with one
of $B$:
\[ A \times B = \{ (x,y) \mid x \in A, y \in B \} \]
%- - - - - - - - - - - - - - - - - - - - - - - - - - - - - - - - - - - - - - - - 
\subsection*{Powersets}

The powerset of a set $S$, denoted $\mathcal{P}(S)$ or $2^S$,  is the set of
all subsets of $S$.

For example:

\begin{align*}
S &= \{1, 2, 3\} \\
{\cal P}(S) &= \{\emptyset, \{1\}, \{2\}, \{3\}, \{1, 2\}, \{1, 3\}, \{2, 3\}, \{1, 2, 3\}\}
\end{align*}

%- - - - - - - - - - - - - - - - - - - - - - - - - - - - - - - - - - - - - - - - 
\subsection*{Multisets}

A multiset is similar to a set, but an element can occur more than once, for
example: $\{1, 1, 1, 2, 2, 3, 4, 5, 5\}$. We call the number of times an 
element appears in a multiset that element's \emph{multiplicity}.

%- - - - - - - - - - - - - - - - - - - - - - - - - - - - - - - - - - - - - - - - 
\subsection*{Predicates and Relations}

Informally, a predicate $\mathcal{P}$\footnote{Confusingly, $\mathcal{P}$ is
used as the name of a predicate and to denote the powerset. You will have to
infer from the context which is meant.} over a set $A$ denotes a property that
elements of set $A$ may have. We write $\mathcal{P}(x)$, with $x \in A$ to denote that
element $x$ has the property. Formally, a predicate $\mathcal{P}$ over a set
$A$ is just a subset of $A$; it is the subset of all elements of $A$ that have
the property. Hence, $\mathcal{P}(x)$ is just an alternative notation for $x \in \mathcal{P}$.

%- - - - - - - - - - - - - - - - - - - - - - - - - - - - - - - - - - - - - - -
\subsection*{Binary Relations}

A binary relation $\mathcal{R}$ over sets $A$ and $B$ is a predicate over $A
\times B$. We say that $x \in A$ is related to $y \in B$ when
$\mathcal{R}(x,y)$ holds. Sometimes, we also write $x\mathcal{R}y$ instead of
$\mathcal{R}(x,y)$.

%- - - - - - - - - - - - - - - - - - - - - - - - - - - - - - - - - - - - - - -
\subsection*{Partial and Total functions}

A function $f: A \mapsto B$ is a binary relation on $A \times B$ where
we write $f(x) = y$ rather than $f(x,y)$ to denote that $x \in A$ and $y \in B$
are related. Moreover, a function satisifies the \emph{functional dependency}
property:
\[ \forall x \in A, \forall y_1, y_2 \in B: f(x) = y_1 \wedge f(x) = y_2 \Rightarrow y_1 = y_2 \]

We say that a function $f$ is \emph{total} iff:
\[ \forall x \in A, \exists y \in B: f(x) = y \]
If a function is not total, it is called \emph{partial}. Usually, when we do
not specify that a function is partial, we implicitly assume that it is total.

%- - - - - - - - - - - - - - - - - - - - - - - - - - - - - - - - - - - - - - -
\subsection*{Equivalence Relations}

A binary relation $\mathcal{R}$ on $S \times S$ is an equivalence relation if and only if
it satisfies three properties:
\begin{center}
\begin{tabular}{r@{\hspace{1cm}}l}
reflexivity  & $\forall x \in S: x\mathcal{R}x$ \\
symmetry     & $\forall x, y \in S: x\mathcal{R}y \Leftrightarrow y\mathcal{R}x$ \\
transitivity & $\forall x, y, z \in S: x\mathcal{R}y \wedge y\mathcal{R}z \Rightarrow x\mathcal{R}z$ 
\end{tabular}
\end{center}

%- - - - - - - - - - - - - - - - - - - - - - - - - - - - - - - - - - - - - - -
\subsection*{Order Relations}

A binary relation $\mathcal{R}$ on $S \times S$ is a \em{partial} order on $S$ if and only if it
has the following three properties:
\begin{center}
\begin{tabular}{r@{\hspace{1cm}}l}
reflexivity  & $\forall x \in S: x\mathcal{R}x$ \\
antisymmetry     & $\forall x, y \in S: x\mathcal{R}y \wedge y\mathcal{R}x \Rightarrow x = y$ \\
transitivity & $\forall x, y, z \in S: x\mathcal{R}y \wedge y\mathcal{R}z \Rightarrow x\mathcal{R}z$ 
\end{tabular}
\end{center}

A partial order $\mathcal{R}$ on $S$ is a \em{total} order if and only if also satisfies a fourth
property:
\begin{center}
\begin{tabular}{r@{\hspace{1cm}}l}
connexity& $\forall x, y \in S: x\mathcal{R}y \vee y\mathcal{R}x$ 
\end{tabular}
\end{center}
Observe that connexity generalizes reflexivity.

%- - - - - - - - - - - - - - - - - - - - - - - - - - - - - - - - - - - - - - -
\subsection*{Transitive closure}

The transitive closure of a binary relation $\mathcal{R}$ on $S \times S$,
denoted $\mathcal{R}^+$, is
the smallest relation on $S \times S$ that contains $\mathcal{R}$ and that 
has the transitivity property.

For example:

\begin{align*}
S &= \{1, 2, 3\} \\
\mathcal{R} &= \{(1, 2), (2, 3)\} \\
\mathcal{R}^+ &= \{(1, 2), (2, 3), (1, 3)\}
\end{align*}

%-------------------------------------------------------------------------------
\section*{Proof Techniques}

%- - - - - - - - - - - - - - - - - - - - - - - - - - - - - - - - - - - - - - -
\subsection*{Weak Induction}

Weak induction is a proof technique that establishes $\forall n \in \mathbb{N}:
\mathcal{P}(n)$, with $\mathcal{P}$ some predicate on $\mathbb{N}$.
It requires proving two simpler statements:
\begin{itemize}
\item[(a)] $\mathcal{P}(0)$, and
\item[(b)] $\forall n \in \mathbb{N}: \mathcal{P}(n) \Rightarrow \mathcal{P}(n+1)$.
\end{itemize}

%- - - - - - - - - - - - - - - - - - - - - - - - - - - - - - - - - - - - - - -
\subsection*{Strong Induction}

Strong induction is a proof technique derived from weak induction that also
establishes $\forall n \in \mathbb{N}: \mathcal{P}(n)$, with $\mathcal{P}$ some
predicate on $\mathbb{N}$. It requires proving two simpler statements, the second
of which differs from weak induction:
\begin{itemize}
\item[(a)] $\mathcal{P}(0)$, and
\item[(b')] $\forall n \in \mathbb{N}: (\forall m \leq n: \mathcal{P}(m)) \Rightarrow \mathcal{P}(n+1)$.
\end{itemize}
