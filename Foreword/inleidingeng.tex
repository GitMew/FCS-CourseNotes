\chapter*{Foreword}

Algorithms are at the heart of computer science. Effective algorithms
for concrete problems are always based on a good choice of the
abstraction and a formal insight in that abstraction. Apparently, {\em
graphs} offer an often useful abstraction context, so we start this
course with an introduction to graph theory: basic concepts, a few
theorems, and some algorithms that are based on the theorems. Graphs
will be useful in the later chapters. After graph theory follow
chapters on {\em languages}: we build up a hierarchy of languages,
the necessary machinery to decide those languages, and the formalisms
to specify them. We pay particular attention to non-decidable
languages, because they define the limits of what can be achieved by
means of an algorithm. The course notes end with an introduction to
complexity theory: one of the central problems of computer science is
defined precisely, i.e. the question whether $\P$ equals
$\NP$. Besides time complexity, we also discuss space complexity.


This text has evolved over time. It was originally written by Bart Demoen in
Dutch and later tranlated to English. The original Dutch text was created out
of notes from his earlier courses, in particular
\begin{verse}
* {\em Fundamenten voor de Informatica} 1$^{ste}$ Bachelor
Informatica, written with K. De Kimpe in 1997,

* {\em Automaten en Berekenbaarheid} 3$^{de}$ Bachelor Informatica,
written in 2004.
\end{verse}
Tom Schrijvers has made many improvements in terms of style, clarity and level
of detail. In terms of content, his most recent and substantial change is the
introduction of more direct mappings between regular expressions and
deterministic finite automata.

Sources and extra material are mentioned at the end of a chapter, the
end of a subject, or the end of the course notes. On purpose, there
are no exercises in this text, but often, the student is invited to
do certain things themselves.

\section*{Acknowledgments}

Many thanks to Marton Bognar for helping with the revision of the 
English course notes in 2019.

We are also grateful to 
Tom Bury, 
Thomas De Backer, 
Yevhen Tsyba,
Nicholas Wellens,
Matthias van der Hallen,
Raf Hermans,
Yandi Liu,
Inias Peeters,
Bo Kleynen
and 
Antoine Van Luchem
for reporting typographical and other errors in this text.

