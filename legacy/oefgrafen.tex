\section{Oefeningen}
\begin{enumerate}

\item Welke van de volgende grafen hebben een Euleriaanse kring?
(Welke hebben een Euleriaans pad?)
\begin{enumerate}
\item $ $ \\
\epsfig{file=oef-euler1.eps, width=8cm}
\item $ $ \\
\epsfig{file=oef-euler2.eps, width=8cm}
\item $ $ \\
\epsfig{file=oef-euler3.eps, width=8cm}
\end{enumerate}
\item Hoeveel bogen hebben $K_n$ en $K_{m,n}$?
\item Beschrijf de  buurmatrix van $K_n$.
\item Geef de incidentiematrix van $K_4$.
\item Beschrijf de buurmatrix van $K_{m,n}$.
\item Geef de incidentiematrix van $K_{4,2}$.
\item Wat is het aantal paden van lengte 64 tussen 2 verschillende 
punten in de graaf $K_{123}$.
%Met de buurmatrix, geeft dit aanleiding tot een schone recursiebetrekking.
\item Geef een voorbeeld van twee grafen die evenveel bogen en evenveel 
knopen hebben en die niet isomorf zijn.
\item Gegeven een enkelvoudig samenhangende graaf $G(V,E)$ met 
$\#V=123$ en $\#E=364$. Toon aan dat $G$ geen vlakke graaf is. 

\item Bewijs dat $K_{3,3}$ geen deelgraaf bevat die homeomorf is
met $K_{5}$, noch omgekeerd.

\item Is de duale van de duale de oorspronkelijke?

\item Welke eigenschappen hebben een graaf en zijn duale gemeen?

\item Los het postman probleem op.

\item Welke eigenschappen blijven bewaard onder een rijreductie?

\item Wat is de minimale $k$ waarvoor $K_{n,m}$ een $k$-kleuring heeft?

\item Voor welke $n$ en $m$ heeft $K_{n,m}$ een Euleriaans pad/kring?

\item In de stelling over de $5$-kleuring is er een voorwaarde te
streng, welke?

\item In de stelling \ref{euler} is elke voorwaarde nodig; laat dat
zien door tegenvoorbeelden. Geldt de omgekeerde stelling ?

\item Noem nog eigenschappen die invariant zijn onder isomorfisme.

\item % Ken je eigenschappen die niet invariant zijn onder isomorfisme?
  Als een vlakke graaf (i.e. een graaf die in het vlak kan getekend
  worden zonder dat de bogen elkaar snijden) op meerdere manieren kan
  getekend worden in het vlak
% (eventueel verwijzen naar tekening)
  zijn dan de dualen horende bij die verschillende tekeningen steeds
  isomorfe grafen? M.a.w. heeft het zin te spreken over ``de'' duale
  van een vlakke graaf?

\item Bewijs dat elke tweeledige graaf een $2$-kleuring heeft.

\item Bewijs dat elke graaf die een $2$-kleuring heeft, tweeledig
is.

\item Voer Dijkstra's algoritme uit op een paar voorbeeldgrafen

\item Beschrijf infix en postfix als een variante op diepte of
breedte eerst

\item Geeft het volgende algoritme een mob $T$? eindigt het? (G is een
  samenhangende graaf)
\begin{itemize}
\item
init: $T = G$
\item
stop?: indien er $(n-1)$ bogen zijn, stop
\item
kies boog: verwijder langste boog in een kring; ga naar stop?
\end{itemize}

\item Bedenk een gulzig algoritme en bewijs of het optimaal is
(vb wissel)

\item Bewijs dat er nooit meer dan twee elementen $(a,n) \in S$
met maximale $n$ kunnen zijn in \textbf{prefix orde} voor binaire bomen

\item Pas diepte-eerst en breedte-eerst aan zodanig dat enkel
knopen tot een bepaalde geven diepte $d$ behandeld worden. Je algoritme
moet ook eindigen voor oneindige bomen.

\item In de definitie van transportnetwerk, mag daar conditie 4
vervangen worden door: ``vanuit de bron is elke knoop bereikbaar langs
gerichte bogen''? of door ``vanuit de bron is elke knoop bereikbaar
langs ongerichte bogen''?

\end{enumerate}

